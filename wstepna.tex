%%%%%%%%%%%%%%%%%%%%%%%%%%%%%%%%%%%%%%%%%
% Short Sectioned Assignment
% LaTeX Template
% Version 1.0 (5/5/12)
%
% This template has been downloaded from:
% http://www.LaTeXTemplates.com
%
% Original author:
% Frits Wenneker (http://www.howtotex.com)
%
% License:
% CC BY-NC-SA 3.0 (http://creativecommons.org/licenses/by-nc-sa/3.0/)
%
%%%%%%%%%%%%%%%%%%%%%%%%%%%%%%%%%%%%%%%%%

%----------------------------------------------------------------------------------------
%	PACKAGES AND OTHER DOCUMENT CONFIGURATIONS
%----------------------------------------------------------------------------------------

\documentclass[paper=a4, fontsize=11pt]{scrartcl} % A4 paper and 11pt font size

\usepackage[utf8]{inputenc}
\usepackage[T1]{fontenc} % Use 8-bit encoding that has 256 glyphs
\usepackage{fourier} % Use the Adobe Utopia font for the document - comment this line to return to the LaTeX default
\usepackage[english]{babel} % English language/hyphenation
\usepackage{amsmath,amsfonts,amsthm} % Math packages

\usepackage{lipsum} % Used for inserting dummy 'Lorem ipsum' text into the template

\usepackage{sectsty} % Allows customizing section commands
\allsectionsfont{\centering \normalfont\scshape} % Make all sections centered, the default font and small caps

\usepackage{fancyhdr} % Custom headers and footers
\pagestyle{fancyplain} % Makes all pages in the document conform to the custom headers and footers
\fancyhead{} % No page header - if you want one, create it in the same way as the footers below
\fancyfoot[L]{} % Empty left footer
\fancyfoot[C]{} % Empty center footer
\fancyfoot[R]{\thepage} % Page numbering for right footer
\renewcommand{\headrulewidth}{0pt} % Remove header underlines
\renewcommand{\footrulewidth}{0pt} % Remove footer underlines
\setlength{\headheight}{13.6pt} % Customize the height of the header

\numberwithin{equation}{section} % Number equations within sections (i.e. 1.1, 1.2, 2.1, 2.2 instead of 1, 2, 3, 4)
\numberwithin{figure}{section} % Number figures within sections (i.e. 1.1, 1.2, 2.1, 2.2 instead of 1, 2, 3, 4)
\numberwithin{table}{section} % Number tables within sections (i.e. 1.1, 1.2, 2.1, 2.2 instead of 1, 2, 3, 4)

\setlength\parindent{0pt} % Removes all indentation from paragraphs - comment this line for an assignment with lots of text

%----------------------------------------------------------------------------------------
%	TITLE SECTION
%----------------------------------------------------------------------------------------

\newcommand{\horrule}[1]{\rule{\linewidth}{#1}} % Create horizontal rule command with 1 argument of height

\title{	
\normalfont \normalsize 
\textsc{Politechnika Warszawska, wydział EiTI} \\ [25pt] % Your university, school and/or department name(s)
\horrule{0.5pt} \\[0.4cm] % Thin top horizontal rule
\huge Dokumenatcja Wstępna \\ % The assignment title
\large Implementacja i porównanie algorytmów odkrywania wzorców sekwencyjnych SPADE i GSP \\
\horrule{2pt} \\[0.5cm] % Thick bottom horizontal rule
}

\author{Tomasz Pęksa, Paweł Stiasny} % Your name

\date{\normalsize13.12.2015} % Today's date or a custom date

\begin{document}

\maketitle % Print the title

%----------------------------------------------------------------------------------------
%	PROBLEM 1
%----------------------------------------------------------------------------------------

\section{Defnicja zadania}

\subsection{Reguły sekwencyjne}

Problem odkrywania wzorców sekwencyjnych możemy opisać następująco: Zdefiniujmy
\( \mathcal I = \{ i_1, i_2, \ldots, i_m \} \) jako alfabet {\em przedmiotów}
({\em items}).  {\em Zdarzenie} ({\em event}) to niepusty zbiór przedmiotów \(
( i_1i_2\ldots i_k ) \) oraz czas jego wystąpienia.  {\em Sekwencja} ({\em
sequence}) to uporządkowana lista zdarzeń \( ( \alpha_1 \rightarrow \alpha_2 \rightarrow \cdots \rightarrow \alpha_q ) \).
Mówimy, że sekwencja \( \alpha \) {\em zawiera} sekwencję \( \beta \) jeśli
istnieje mapowanie \( f \) takie, że:
\begin{enumerate}
    \item Dla każdego zdarzenia w \( \alpha \): \( \alpha_i \subseteq f(\alpha_i) \),
    \item \( f(\alpha_i) \) zachodzi przed \( f(\alpha_j) \) jesli \( \alpha_i \)
        zachodzi przed \( \alpha_j \).
\end{enumerate}

Istnieje baza danych \( \mathcal D \) zdefiniowana jako zbiór sekwencji wraz
z odpowiadającymi im identyfikatorami {\em sid}.  Identyfikatorem każdego ze
zdarzeń w sekwencji jest jego czas wystąpienia {\em eid}.

{\em Wsparcie} ({\em support}) sekwencji rozumiemy jako liczbę wszystkich
sekwencji w bazie \( \mathcal D \) które zawierają tę sekwencję.  Sekwencja
jest {\em częsta}, jeśli jej wsparcie jest wyższe niż wybrana przez użytkownika
wartość {\em min\_sup}.  Zadanie odkrywania wzorców sekwencyjnych w bazie \(
\mathcal D \) przy ustalonej wartości {\em min\_sup} definiujemy jako zadanie
znalezienia wszystkich sekwencji częstych w tej bazie.

Przez {\em reguły sekwencyjne} rozumiemy wyrażenia postaci \( \alpha
\Rightarrow \beta \), gdzie \(\beta \) zawiera \( \alpha \).  Znając wsparcia
rozważanych sekwencji możemy wyznaczyć {\em zaufanie} ({\em confidence})
reguły jako \( \text{conf}(\alpha \Rightarrow \beta) = \frac{\text{support}(\alpha)}{\text{support}(\beta)}\).
Zadanie odkrywania reguł sekwencyjnych polega na znalezieniu wszystkich reguł
takich, że sekwencje są częste i zaufanie jest większe niż wybrana przez
użytkownika wartość {\em min\_conf}.

\subsection{Algorytm SPADE}
W odróżnieniu od GSP, SPADE wymaga jedynie pojedynczego odczytu bazy danych.
Algorytm ten korzysta z wydajnej pośredniej reprezentacji zbioru sekwencji
częstych.

\subsubsection{Struktury danych}
Budowany jest graf reprezentujący podkratę relacji zawierania sekwencji.  W
kracie tej poprzednikami \( k \)-elementowej sekwencji są dwie \( k-1 \)
-elemetnowe sekwencje o wspólnym, \( k-2 \)-elementowym prefixie.  Następniki
są generowane przez {\em złączenie czasowe} dwóch takich poprzedników.  Węzeł
grafu przechowuje sekwencję oraz listę par (sid, eid) odpowiadających
wystąpieniom ostatniego elementu sekwencji.  Autor dowodzi, że taka
reprezenacja jest wystarczająca do wygenerowania wszystkich sekwencji częstych.

\subsubsection{Złączenie czasowe}
Mając dwie sekwencje długości \( k-1 \) i odpowiadające im listy
identyfikatorów algorytm generuje nowy zbiór sekwencji długości \( k \)
o wspólnym prefixie długości \( k-2 \).  W zależności od rodzaju złączanych
sekwencji wymaga to przejrzenia list identyfikatorów pod kątem występowania
zdarzeń w tym samym czasie i/lub następstwa zdarzeń w czasie.  Sekwencje
o wsparciu mniejszym niż {\em min\_sup} są odrzucane.

\subsubsection{Zliczanie wsparcia}
Przy przyjętych strukturach danych obliczenie wsparcia sekwencji sprowadza się
do zliczenia unikatowych identyfikatorów sekwencji {\em sid} w liście
identyfikatorów interesującego nas węzła.

\subsubsection{Przebieg algorytmu}
Sekwencje jednoelementowe są wczytywane z bazy danych.  Możliwe jest też
wczytanie wstępnie przetworzonych sekwencji dwuelementowych.  Następnie algorytm
przeszukując wgłab lub wszerz rekurencyjnie buduje graf w sposób opisany powyżej.

\subsection{Algorytm GSP}
\subsection{Porównanie}

\section{Projekt realizacji}
\subsection{Reguły sekwencyjne}
\subsection{Algorytm SPADE}
\subsection{Algorytm GSP}
\subsection{Porównanie}


\end{document}

